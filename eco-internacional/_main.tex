\documentclass[]{article}
\usepackage{lmodern}
\usepackage{amssymb,amsmath}
\usepackage{ifxetex,ifluatex}
\usepackage{fixltx2e} % provides \textsubscript
\ifnum 0\ifxetex 1\fi\ifluatex 1\fi=0 % if pdftex
  \usepackage[T1]{fontenc}
  \usepackage[utf8]{inputenc}
\else % if luatex or xelatex
  \ifxetex
    \usepackage{mathspec}
  \else
    \usepackage{fontspec}
  \fi
  \defaultfontfeatures{Ligatures=TeX,Scale=MatchLowercase}
\fi
% use upquote if available, for straight quotes in verbatim environments
\IfFileExists{upquote.sty}{\usepackage{upquote}}{}
% use microtype if available
\IfFileExists{microtype.sty}{%
\usepackage{microtype}
\UseMicrotypeSet[protrusion]{basicmath} % disable protrusion for tt fonts
}{}
\usepackage[margin=1in]{geometry}
\usepackage{hyperref}
\hypersetup{unicode=true,
            pdfborder={0 0 0},
            breaklinks=true}
\urlstyle{same}  % don't use monospace font for urls
\usepackage{longtable,booktabs}
\usepackage{graphicx,grffile}
\makeatletter
\def\maxwidth{\ifdim\Gin@nat@width>\linewidth\linewidth\else\Gin@nat@width\fi}
\def\maxheight{\ifdim\Gin@nat@height>\textheight\textheight\else\Gin@nat@height\fi}
\makeatother
% Scale images if necessary, so that they will not overflow the page
% margins by default, and it is still possible to overwrite the defaults
% using explicit options in \includegraphics[width, height, ...]{}
\setkeys{Gin}{width=\maxwidth,height=\maxheight,keepaspectratio}
\IfFileExists{parskip.sty}{%
\usepackage{parskip}
}{% else
\setlength{\parindent}{0pt}
\setlength{\parskip}{6pt plus 2pt minus 1pt}
}
\setlength{\emergencystretch}{3em}  % prevent overfull lines
\providecommand{\tightlist}{%
  \setlength{\itemsep}{0pt}\setlength{\parskip}{0pt}}
\setcounter{secnumdepth}{5}
% Redefines (sub)paragraphs to behave more like sections
\ifx\paragraph\undefined\else
\let\oldparagraph\paragraph
\renewcommand{\paragraph}[1]{\oldparagraph{#1}\mbox{}}
\fi
\ifx\subparagraph\undefined\else
\let\oldsubparagraph\subparagraph
\renewcommand{\subparagraph}[1]{\oldsubparagraph{#1}\mbox{}}
\fi

%%% Use protect on footnotes to avoid problems with footnotes in titles
\let\rmarkdownfootnote\footnote%
\def\footnote{\protect\rmarkdownfootnote}

%%% Change title format to be more compact
\usepackage{titling}

% Create subtitle command for use in maketitle
\newcommand{\subtitle}[1]{
  \posttitle{
    \begin{center}\large#1\end{center}
    }
}

\setlength{\droptitle}{-2em}
  \title{}
  \pretitle{\vspace{\droptitle}}
  \posttitle{}
  \author{}
  \preauthor{}\postauthor{}
  \date{}
  \predate{}\postdate{}


\begin{document}

{
\setcounter{tocdepth}{2}
\tableofcontents
}
\section{Introducción}\label{introduccion}

En este documento se exponen los modelos utilizados en la clase de
Economía Interacional II. Se omiten alunos detalles algebraicos y en
todos ellos se utiliza un agente representativo.

\subsection{Convenciones}\label{convenciones}

A lo largo de los caítulos se utilizan las siguientes convenciones:

\begin{itemize}
\tightlist
\item
  Se utiliza \(t\) para denotar al periodo presente.
\item
  Se utiliza \(s\) para denotar un periodo arbitrario.
\item
  Se utiliza \(h\) para denotar un número de periodos arbitrario.
\end{itemize}

Además se utiliza la siguiente notación:

\[
\sum_{s = t}^\infty x_s = \sum_{s \geq t} x_s
\] Para cualquier variable \(x_s\).

\section{Modelo con un solo bien y con
dotación}\label{modelo-con-un-solo-bien-y-con-dotacion}

\subsection{Problema de optimización}\label{problema-de-optimizacion}

En este modelo en la economía existe un solo bien. En cada periodo la
cantidad que consume el agente representativo se denota con \(c_s\) para
\(s\geq t\). El objetivo del agente representativo es

\[
\max_{c_s, b_{s+1}, s \geq t} \{ \sum_{s\geq t} \beta^{s-t}\ln(c_s):  c_s + b_{s+1} = y_s + (1+r_{s-1})b_s, s \geq t\}
\] En donde \(b_s\) denota la cantidad del bien que el agente
representativo ahorra en el periodo \(s\), \(y_s\) la dotación del bien
en el periodo \(s\) y \(r_s\) la tasa de interés del periodo \(s\).
Además, en este problema están dados \(\{y_s, r_{s-1}\}_{s\geq t}\) y
\(b_t\).

Por otro lado, se define

\[
T_s = y_s- c_s
\] La balanza comercial del país en el periodo \(s\) y además

\[
CA_s = b_{s+1} - b_s = T_s + r_{s-1}b_s
\] La cuenta corriente del país en el periodo \(s\). La identidad
anterior surge de la restricción presupuestal del agente representativo.

\subsection{Condiciones de primer
orden}\label{condiciones-de-primer-orden}

De las condiciones de primer orden se obtiene la siguiente condición de
optimalidad, conocida como la ecuación de euler,

\[
c_{s+1} = \beta(1+r_s)c_s
\] Para cada \(s \geq t\).

\subsection{Restricción presupuestal}\label{restriccion-presupuestal}

Para \(s \geq t\) y \(j \geq s\) definimos \(\Delta_j\) como
\(\Delta_j = 1\) si \(j = s\) y

\[
\Delta_j = \prod_{k = s}^{j-1} \frac{1}{1+r_k}
\] Para \(j \geq s + 1\). Iterando hacia adeltante con la restricción
presupuestal del agente representativo a partir del periodo \(s\) se
obtiene que en el óptimo,

\[
\sum_{j\geq s} \Delta_jT_j = -(1+r_{s-1})b_s 
\] O bien,

\[
\begin{equation}
\sum_{j\geq s} \Delta_jc_j = (1+r_{s-1})b_s  + \sum_{j\geq s} \Delta_jy_j
\label{eq:rp}
\end{equation}  
\]

\emph{En particular} para el periodo \(t\) se verifica que,

\[
\sum_{s \geq t} \Delta_s c_s = (1+r_{t-1})b_t + \sum_{s \geq t} \Delta_s y_s
\]b\_ Con lo anterior estamos listos para establecer resultados a partir
de supuestos sobre las variables dadas. Pero antes enunciemos un
\textbf{resultado fundamental}.

\subsection{Solución}\label{solucion}

De la ecuación de euler sabemos que \(c_{s+1} = \beta(1+r_s)c_s\) para
cada \(s \geq t\). Luego, para cada periodo \(s \geq t\) y \(j \geq s\)
se verfifica que,

\[
\begin{align}
\Delta_jc_j &= \Delta_j \beta(1+r_{j-1}) c_{j-1} \\
&= \beta \Delta_{j-1}c_{j-1}
\end{align}
\] Iterando lo anterior hasta el periodo \(s\) obtenemos que,

\[
\Delta_j c_j = \beta^{j-s} \Delta_s c_s
\]

Por lo tanto,

\[
\sum_{j \geq s} \Delta_j c_j = \sum_{j\geq s} \beta^{j-s} \Delta_s c_s = \frac{\Delta_s c_s}{1-\beta}
\]

Sustituyendo la ecuación anterior en \eqref{eq:rp} obtenemos que para
cualquier \(s \geq t\),

\[
\begin{align}
\Delta_s c_s &= (1-\beta)((1+r_{s-1})b_s + \sum_{j\geq s}\Delta_s y_s) \\
&= (1-\beta)(1+r_{s-1})b_s + (1-\beta)\sum_{j\geq s}\Delta_s y_s \\
&= (1-\beta)(1+r_{s-1})b_s + \hat{y_s}
\end{align}
\]

Donde \(\hat{y_s} = (1- \beta)\sum_{j\geq s} \beta^{j-s}y_j\) es el
valor permanente de \(y\) en el periodo \(s\). Como \(\Delta_s = 1\) en
este caso, esto resuelve el problema pues el consumo para cada periodo
\(s \geq t\) está dado por,

\[
c_s = (1-\beta)(1+r_{s-1})b_s +\hat{y_s}
\]

Y todos los valores del lado derecho de la igualdad son conocidos.
\emph{En particular}, para \(s = t\) se tiene que,

\[
\begin{equation}
c_t = (1-\beta)(1+r_{t-1})b_t + \hat{y_t}
\label{eq:cons}
\end{equation}
\]

Este resultado es bastante general pues \emph{se obtuvo a partir de la
condición de optimalidad y la restricción presupuestal} \textbf{sin
hacer supuestos sobre los parámetros del modelo}.

\subsection{Supuestos habituales}\label{supuestos-habituales}

\subsubsection{Tasa de interés
constante}\label{tasa-de-interes-constante}

En este caso, se supone que \(r_{s-1} = r^*\) para cada \(s \geq t\).
Por lo que,

\[
\Delta_s = \left(\frac{1}{1 + r^*}\right)^{s-t}
\]

\subsubsection{Consumo constante}\label{consumo-constante}

Se asume, además de la tasa de interés contante que,
\(\beta(1+r^*) = 1\). En este caso, \(c = c_{s+1} = c_{s}\) para cada
\(s \geq t\) (a partir de la ecuación de euler). Así, se verifica que

\[
\Delta_s = \beta^{s-t}
\] Con esto,

\[
\hat{y_s} = (1-\beta)\sum_{j \geq s}\beta^{j-s}y_j
\] Obsérvese que \((1-\beta)(1+r^*) = 1 + r^* - \beta(1+r^*) = r^*\).
Por lo tanto, a partir de \eqref{eq:cons},

\[
c = c_s = r^*b_s + \hat{y_s}
\]

De lo anterior se desprende que la cuenta corriente en el periodo \(s\)
está determinada por,

\[
CA_s = T_s + r^*b_s =  y_s - c_s + r^*b_s = y_s - \hat{y_s} 
\]

\subsection{Ejemplo}\label{ejemplo}

Supongamos que \(r_{s-1} = r^*\) para cada \(s \geq t\) y que
\(\beta*(1+r^*) = 1\). Además, \(y_s = y\) para cada \(s\geq t\) y
\(b_t = 0\). Los dos primeros supuestos establecen que
\(c = r^*b_t + \hat{y_t}\). Del tercer supuesto se obtiene que

\[
\hat{y_t} = (1-\beta)\sum_{s\geq t} \beta^{s-t} y_s = (1-\beta)y \sum_{s\geq t} \beta^{s-t} = y
\]

Por lo tanto, usando además el último supuesto, se tiene que
\(c = c_s = y\) para cada \(s \geq t\). Con eso, claramente
\(T_s = CA_s = 0\) para cada \(s\geq t\).

\subsubsection{Choque}\label{choque}

Supongamos ahora que hay un choque en la dotación durante \(h\)
periodos. Formalmente, ahora \(y_s = \lambda y\) com \(\lambda < 1\) si
\(t \leq s \leq t+h-1\) y \(y_s = y\) para cada \(s \geq t+h\) y se
mantiene el supuesto de que \(b_t = 0\). Recordemos que, en general, se
sigue sosteniendo que el consumo es constante y que,

\[
c = r^*b_t + \hat{y_t}
\]

Por lo que basta encontrar el valor permanente de la dotación en \(t\).
En general, si \(t \leq s \leq t+h-1\) entonces,

\[
\begin{align*}
\hat{y_s} &= (1-\beta)\sum_{j \geq s} \beta^{j-s}y_j \\
&= (1-\beta)\left(\sum_{j = s}^{t+h-1}\beta^{j-s}\lambda y + \sum_{j \geq t+h} \beta^{j-s} y\right) \\
& = (1-\beta)\left(\lambda y \sum_{j = s}^{t+h-1}\beta^{j-s} + y \sum_{j \geq t+h} \beta^{j-s }  \right) \\
& = (1-\beta)\left(\frac{1-\beta^{h+(t-s)}}{1-\beta}\lambda y + \frac{\beta^{h+(t-s)}}{1-\beta}y\right) \\
& = \left(1-\beta^{h+(t-s)}\right)\lambda y + \beta^{h+(t-s)} y
\end{align*}
\]

Y si \(s \geq t\) entonces \(\hat{y_s} = y\). Por lo tanto,
\(\hat{y_t} = (1-\beta^h)\lambda y + \beta^h y\) y usando que
\(\beta_t = 0\), se verifica que
\(c = (1-\beta^h)\lambda y + \beta^h y\). Con esto, para
\(t \leq s \leq t+h-1\) se sostiente que,

\[
T_s = y_s - c_s = \lambda y - (1-\beta^h)\lambda y - \beta^h y = \beta^h(\lambda - 1)y < 0
\]

Por lo que durante el choque, se importan bienes. En cambio, para
\(s \geq t+h\) se tiene que ,

\[
\begin{align*}
T_s &= y_s - c_s \\
& = y -(1-\beta^h)\lambda y - \beta^h y \\
& =  (1 - \lambda + \lambda \beta^h  - \beta^h)y \\
& = (1-\lambda)(1-\beta^h)y > 0
\end{align*}
\]

Por lo que tras el choque se exportan bienes. Por el lado de la cuenta
corriente, si \(t \leq s \leq t+h-1\) entonces

\[
\begin{align*}
CA_s & = y_s - \hat{y_s} \\
& = \lambda y - \left(1-\beta^{h+(t-s)}\right)\lambda y - \beta^{h+(t-s)} y \\
& = \beta^{h +(t-s)}(\lambda-1)y < 0 
\end{align*}
\]

Y si \(s \geq t+h\) entonces claramente \(CA_s = 0\). Con ,o anterior
podemos determinar los niveles de ahorro. Recordemos que
\(CA_s = T_s + r^*b_s\). Así, para \(t \leq s \leq t+h-1\),

\[
\beta^{h +(t-s)}(\lambda-1)y = CA_s = T_s + r^*b_s = \beta^h(\lambda - 1)y + r^*b_s
\] Por lo tanto, dado que \(\frac{1}{r^*} = 1-\beta\)

\[
b_s = (1-\beta)\beta^h\left(\beta^{t-s} - 1\right)(\lambda-1)y < 0
\]

Por otro lado, si \(s \geq t+h\) entonces,

\[
0 = CA_s = T_s + r^* b_s = (1-\lambda)(1-\beta^h)y + r^* b_s
\]

Así, en este caso,

\[
b_s = (1-\beta)(1-\beta^h)(\lambda-1)y < 0
\] Por lo que en este choque, siempre hay deuda.

\subsubsection{Choque permanente}\label{choque-permanente}

Supongamos el choque anterior pero de manera permanente. En este caso,
\(h \rightarrow \infty\), por lo que \(b_s, CA_s, T_s \rightarrow 0\) y
\(c \rightarrow \lambda y\).

\section{Modelo con un solo bien, dotación y
gobierno}\label{modelo-con-un-solo-bien-dotacion-y-gobierno}

\subsection{Gobierno}\label{gobierno}

En este caso el gobierno compra \(g_s\) cantidad del bien en la economía
para cada \(s \geq t\); es decir, la senda de compras del gobierno
\(\{g_s\}_{s \geq t}\) está dada. Para financiar estas compras, el
gobierno establece un impuesto de suma fija \(\tau_s\) para cada
\(s \geq t\) y además se endeuda a la misma tasa a la que el agente
representativo presta bonos. Se denota como \(d_{s-1}\) para
\(s \geq t\) la deuda del gobierno con \(d_t\) dada. Por lo tanto, la
restricción presupuestal del gobierno está dada por

\[
\begin{equation}
g_s = \tau_s + d_{s+1} - (1+r_{s-1})d_s
\label{eq:rpgob}
\end{equation}
\]

Para cada \(s \geq t\). El déficit del gobierno es
\(d_{s+1} -d_s = g_s - \tau_s + r^*d_s\). Además, se verifica que para
cada \(s \geq t\),

\[
\sum_{j \geq s} \Delta_j g_j = -(1 + r_{s-1})d_s + \sum_{j \geq s} \Delta_j \tau_j 
\] Es decir,

\[
\begin{equation}
\hat{g_s} = -(1-\beta)(1+ r_{s-1})d_s + \hat{\tau_s}
\label{eq:rpgobperm}
\end{equation}
\]

Donde \(\hat{g_s}\) y \(\hat{\tau_s}\) son el valor permanente de las
compras del gobierno en \(s\) y el valor permanente de la recaudación
del gobierno en \(s\), respectivamente.

\subsection{Problema de optimización}\label{problema-de-optimizacion-1}

El objetivo del agente representativo es,

\[
\max_{c_s, b_{s+1}, s\geq t} \{ \sum_{s \geq t} \beta^{s-t}(\ln(c_s) + v(g_s)): c_s + \tau_s + b_{s+1} = y_s + (1+r_{s-1})b_s, s \geq t \} 
\]

En este problema están dados \(\{y_s, r_{s-1}, g_s\}_{s\geq t}\),
\(b_t\) y \(d_t\).

Despejando \(\tau_s\) de \eqref{eq:rpgob} y sustituyendo en la restricción
presupuestal del agente representativo obtenemos que,

\[
\begin{equation}
c_s + g_s + b_{s+1} - d_{s+1} = y_s + (1+r_{s-1})(b_s - d_s)
\label{eq:rpgobcons}
\end{equation}
\] Para \(s \geq t\). En este caso, la balanza comercial está dada por,

\[
T_s = y_s - c_s - g_s
\]

Y la cuenta corriente está dada por,

\[
CA_s = (b_{s+1} - d_{s+1}) - (b_s - d_s) = T_s + r_{s-1}(b_s -d_s)
\]

\subsection{Condiciones de primer
orden}\label{condiciones-de-primer-orden-1}

A partir de las condiciones de primer orden llegamos a la misma ecuación
de euler dado que los impuestos son de suma fija, para cada
\(s \geq t\),

\[
c_{s+1} = \beta(1 + r_{s})c_s
\]

\subsection{Restricción presupuestal}\label{restriccion-presupuestal-1}

Iterando hacia adelante \eqref{eq:rpgobcons} llegamos a que para cada
\(s\), en el óptimo,

\[
\sum_{j \geq s} \Delta_j T_j = -(1 + r_{s-1})(b_s-d_s)
\] O bien,

\[
\sum_{j\geq s} \Delta_j c_j = (1 + r_{s-1})(b_s -d_s) + \sum_{j \geq s} \Delta_j y_j - \sum_{j \geq s} \Delta_j g_j
\]

\subsection{Solución}\label{solucion-1}

Como vimos, a partir de la ecuación de euler se verfifica que para cada
\(s \geq t\) y cada \(j \geq s\),

\[
\Delta_j c_j = \beta^{j-s} \Delta_s c_s
\] Por lo que, en equilibrio, el valor del consumo en el periodo \(s\)
está dado por,

\[
c_s = (1-\beta)(1 + r_{s-1})(b_s-d_s) + \hat{y_s} - \hat{g_s}
\]

\subsection{Supuestos habituales}\label{supuestos-habituales-1}

Si \(r_{s-1} = r^*\) y \(\beta(1+r^*) = 1\) entonces,

\[
c_s = r^*(b_s-d_s) + \hat{y_s} - \hat{g_s}
\] Para cada \(s \geq t\). Con esto,

\[
\begin{align}
CA_s &= T_s + r^*(b_s-d_s) \\
&= y_s - c_s - g_s + r^*(b_s-d_s)\\
&= (y_s - \hat{y_s}) - (g_s - \hat{g_s})
\end{align}
\]

\subsection{Ejemplo}\label{ejemplo-1}

Supongamos que \(r_{s-1} = r^*\), \(\beta(1+r^*)=1\), \(y_s = y\),
\(g_s = g\) para cada \(s \geq t\) y que \(b_t - d_t = 0\). De los
primeros dos supuestos obtenemos que el consumo es el mismo en todos los
periodos, esto es \(c = c_{s+1} = c_s\) para cada \(s \geq t\). Del
tercer y cuarto supuesto obtenemos que \(\hat{y_s} = y\) y
\(\hat{g_s} = g\) para cada \(s \geq t\). Por lo tanto, utilizando el
último supuesto,

\[
c = c_t = (1-\beta)(1+r^*)(b_t-d_t) + \hat{y_t} - \hat{g_t} = y-g
\] Claramente \(T_s = CA_s = 0\) para cada \(s \geq t\).

\subsubsection{Choque en el gasto del gobierno
1}\label{choque-en-el-gasto-del-gobierno-1}

Ahora, además de los supuestos anteriores, supongamos que el gobierno
eleva sus compras a partir del periodo \(t\). Es decir, para cada
\(s \geq t\) se tiene que \(g_s = \lambda g\), con \(\lambda > 1\).
Denotamos como \(c_1\) el consumo durante este choque. Luego,
\(\hat{g_s} = \lambda g\); por lo tanto,

\[
c_1 = y - \lambda g = c + (1-\lambda)g < c
\]

Es decir, en este caso se reduce el nivel de consumo. Como el choque es
permanente se sostiene que \(T_s = CA_s = 0\).

\subsubsection{Choque en el gasto del gobierno
2}\label{choque-en-el-gasto-del-gobierno-2}

Ahora supongamos que el choque anterior sucede en sólo un periodo. Es
decir, \(g_s = \lambda g\) para \(\lambda >1\) si \(s = t\) y
\(g_s = g\) si \(s \geq t+1\). En este caso,

\[
\begin{align}
\hat{g_t} &= (1-\beta)\sum_{s\geq t} \beta^{s-t} g_s \\
&=(1-\beta)\left(\lambda g + \sum_{s\geq t+1}\beta^{s-t}g \right) \\
&= (1-\beta)\left(\lambda g + \beta g \sum_{s\geq t+1} \beta^{s-(t+1)}\right) \\
&= (1-\beta)\lambda g + \beta g 
\end{align}
\]

Y \(\hat{g_s} = g\) si \(s \geq t+1\). Si denotamos \(c_2\) el consumo
generado por este llegamos a que,

\[
c_2 = y - (1-\beta)\lambda g - \beta g = c + (1-\beta)(1-\lambda)g < c
\]

Dado que el choque es de un periodo, la balanza comercial y la cuenta
corriente cambian en el tiempo. Así, en este caso,

\[
\begin{align}
T_t &= y_t - c_t - g_t  \\
&= y - (y - (1-\beta)\lambda g - \beta g) - \lambda g \\
&= \beta (1 - \lambda) g < 0
\end{align}
\]

Por lo que en este periodo se importan bienes. Por otro lado, si
\(s \geq t+1\) entonces,

\[
\begin{align}
T_s &= y_s - c_s - g_s \\
& = y - (y - (1-\beta)\lambda g - \beta g) - g \\
& = \beta(\lambda - 1)g > 0
\end{align}
\]

Por lo que en estos periodos se exportan bienes. Para la cuenta
corriente, recordemos que \(b_t - d_t = 0\) por lo que \(CA_t = T_t\).
Por otro lado, \emph{para encontrar el valor de la deuda externa es
necesario establecer supuestos sobre la recaudación}. Supongamos que
\(\tau_s = \tau\) para cada \(s \geq t\) (siempre cobra los mismos
impuestos) y que \(d_t = 0\). Luego, por \eqref{eq:rpgobperm} se tiene que

\[
(1-\beta)\lambda g + \beta g = \hat{g_t} = \tau
\]

Así,

\[
\begin{align}
d_{t+1} &= g_t - \tau_t \\
&= \lambda g - \tau \\
&= \lambda g -(1-\beta)\lambda g - \beta g \\
&= \beta (\lambda -1) g \\
& = -CA_t
\end{align}
\] Por lo que el gobierno se endeuda en \(t\) y \(b_{t+1} = 0\). Si por
otro lado, el presupuesto es balanceado, es decir \(\tau_s = g_s\) para
cada \(s \geq t\) entonces \(d_{t+1} = 0\) y
\(b_{t+1} = \beta (1- \lambda)g < 0\).

\subsubsection{Choque financiero}\label{choque-financiero}

Ahora supongamos que la tasa de interés aumenta en el periodo \(t\) de
forma inesperada. Formalmente, supongamos que
\(\beta(1+r_s) = \frac{1}{\lambda}\) con \(\lambda < 1\) si \(s = t\) y
\(\beta(1+r_{s}) = \beta(1+r^*)=1\) para \(s = t-1\) y \(s \geq t+1\).
Además supongamos que \(y_s = y\) y \(g_s = g\) para \(s \geq t\) y
\(b_t - d_t < 0\). Bajo estas condiciones, para cualquier \(s \geq t+1\)
se cumple que

\[
\begin{align}
\Delta_s &= \prod_{k  = t} ^{s-1} \frac{1}{1+r_k} \\
&= \lambda \beta^{s-t}
\end{align}
\] Por lo tanto,

\[
\begin{align}
\hat{y_t} &= (1-\beta)\sum_{s \geq t} \Delta_s y \\
&= (1-\beta)y(1 + \sum_{s \geq t+1} \lambda \beta^{s-t}) \\
&= (1-\beta)y\left(1 + \frac{\lambda \beta}{1-\beta}\right) \\
&= (1 - \beta)y + \beta \lambda y
\end{align}
\]

Equivalentemente \(\hat{g_t} = (1-\beta)g + \beta \lambda g\). Con esto,
obtenemos que,

\[
\begin{align}
c_t &= (1-\beta)(1+r_{t-1})(b_t-d_t) + \hat{y_t} - \hat{g_t} \\
&= r^*(b_t-d_t) + (1-\beta)(y-g) + \beta \lambda (y-g)
\end{align}
\]

Denotemos por \(c_0\) el valor del consumo en ausencia de choque; es
decir, \(\lambda = 1\). Luego, \(c_0 = r^*(b_t - d_t) + y- g\). Así,

\[
c_t = c_0 + \beta(\lambda - 1)(y-g) < c_0
\]

Tras el choque, se verifica que \(c = c_{s+1} = c_s\) para
\(c \geq t+1\). Usando la ecuación de euler entre \(t\) y \(t+1\)
obtenemos que

\[
\begin{align}
c &= c_{t+1} \\
&= \beta(1 + r_t)c_t \\
&= \frac{c_t}{\lambda}
\end{align}
\] La cuenta corriente en \(t\) está dada por,

\[
\begin{align}
CA_t &= y- c_t - g + r^*(b_t - d_t) \\
& = y - g - (1-\beta)(y-g) - \beta \lambda (y-g) \\
& = \beta(1-\lambda)(y-g) > 0
\end{align}
\]

\section{Modelo con un solo bien, dotación y términos de
intercambio}\label{modelo-con-un-solo-bien-dotacion-y-terminos-de-intercambio}

\subsection{Términos de intercambio}\label{terminos-de-intercambio}

En este modelo se introduce la variable \(\Omega_s\) que denota el
precio de las exportaciones respecto al precio de las importaciones para
cada \(s \geq t\). Se supone un país pequeño por lo que la secuencia
\(\{\Omega_s\}_{s \geq t}\) está dada.

\subsection{Problema de optimización}\label{problema-de-optimizacion-2}

El objetivo del agente representativo es

\[
\max_{c_s, b_{s+1}, s \geq t} \{ \sum_{s\geq t} \beta^{s-t}\ln(c_s):  c_s + b_{s+1} = \Omega_s y_s + (1+r_{s-1})b_s, s \geq t\}
\]

En este problema están dados \(\{y_s, r_{s-1}, \Omega_s\}_{s\geq t}\) y
\(b_t\). Se asume que \(y_s\) se exporta por completo y \(c_s\) se
importa.

Por otro lado, se define

\[
T_s = \Omega_s y_s- c_s
\] La balanza comercial del país en el periodo \(s\) y además

\[
CA_s = b_{s+1} - b_s = T_s + r_{s-1}b_s
\] La cuenta corriente del país en el periodo \(s\). La identidad
anterior surge de la restricción presupuestal del agente representativo.

\subsection{Condiciones de primer
orden}\label{condiciones-de-primer-orden-2}

En este problema se sigue sosteniendo la ecuación de optimalidad de
euler. Para cada \(s \geq t\),

\[
c_{s+1} = \beta(1 + r_{s})c_s
\]

\subsection{Restricción presupuestal}\label{restriccion-presupuestal-2}

Iterando hacia adeltante con la restricción presupuestal del agente
representativo a partir del periodo \(s\) se obtiene que en el óptimo,

\[
\sum_{j\geq s} \Delta_jT_j = -(1+r_{s-1})b_s 
\] O bien,

\[
\begin{equation}
\sum_{j\geq s} \Delta_jc_j = (1+r_{s-1})b_s  + \sum_{j\geq s} \Delta_j\Omega_j y_j
\end{equation}  
\]

\emph{En particular} para el periodo \(t\) se verifica que,

\[
\sum_{s \geq t} \Delta_s c_s = (1+r_{t-1})b_t + \sum_{s \geq t} \Delta_s \Omega_s y_s
\]

\subsection{Solución}\label{solucion-2}

Iterando la ecuación de euler como en el capítulo inicial obetenemos que
para cada \(s \geq t\),

\[
c_s = (1-\beta)(1+r_{s-1})b_s + \widehat{\Omega_s y_s}
\]

Donde \(\hat{\Omega_s y_s}\) es el valor permanente de la dotación en
términos de las importaciones en el periodo \(s\).

\subsection{Ejemplo}\label{ejemplo-2}

Los choques en los términos de intercambio son similares a los choques
en la dotación.

\section{Modelo con un solo bien, gobierno y
producción}\label{modelo-con-un-solo-bien-gobierno-y-produccion}

En este modelo la cantidad del bien disponible en la economía se produce
por lo que \(y_s = A_sf(k_s)\) para cada \(s \geq t\) y donde \(A_s\) es
la productividad en el periodo \(s\) y \(k_s\) es la cantidad de capital
en la economía en el periodo \(s\). Se asume que no hay depreciación del
capital. Además la inversión para cada periodo \(s \geq t\) está dada
por \(i_s = k_{s+1} - k_s\)

\subsection{Problema de optimización}\label{problema-de-optimizacion-3}

El agente representativo busca

\[
\max_{c_s,b_{s+1}, k_{s+1}, s \geq t} \{ \sum_{s \geq t} \beta^{s-t}(\ln(c_s) + v(g_s)):c_s + b_{s+1} + \tau_s + k_{s+1}-k_s = A_sf(k_s) + (1+r_{s-1})b_s, s\geq t\}
\] EL gobierno debe cumplir su restricción presupuestal como lo hemos
visto en secciones anteriores. En este problema están dados
\(\{y_s, r_{s-1}\}_{s\geq t}\), \(b_t\), \(d_t\) y \(k_t\).
Introduciendo la restricción presupuestal del gobierno en la del agente
representativo y la inversión se obtiene que,

\[
\begin{equation}
c_s + g_s + b_{s+1} - d_{s+1} + i_s = A_sf(k_s) + (1+r_{s-1})(b_s - d_s)
\label{eq:rpcapgob}
\end{equation}
\] Así, la balanza comercial está dada por,

\[
T_s = A_s f(k_s) - c_s - g_s - i_s
\] En tanto que la cuenta corriente está dada por,

\[
CA_s = A_sf(k_s) - c_s - g_s -i_s + r_{s-1}(b_s - d_s) = T_s + r_{s-1}(b_s - d_s)
\]

\subsection{Condiciones de primer
orden}\label{condiciones-de-primer-orden-3}

Las condciones de optimalidad de este problema están dadas por

\[
c_{s+1} = \beta(1+r_s)c_s
\]

Para cada \(s \geq t\). Y además,

\[
A_{s+1}f'(k_{s+1}) = r_s
\]

\subsection{Restricción presupuestal}\label{restriccion-presupuestal-3}

Iterando la restricción presupuestal \eqref{eq:rpcapgob} a partir de \(s\)
se obtiene que, en el óptimo, para cada \(s \geq t\),

\[
\sum_{j \geq s} \Delta_j T_j = -(1+ r_{s-1})(b_s-d_s)
\] O bien,

\[
\sum_{j \geq s} \Delta_j c_j = (1 + r_{s-1})(b_s - d_s) + \sum_{j \geq s}\Delta_j(A_j f(k_j) -g_j-i_j) 
\]

\emph{En particular} para el presente se verifica que,

\[
\sum_{s \geq t} \Delta_s c_s = (1 + r_{t-1})(b_t - d_t) + \sum_{s \geq t} \Delta_s (A_sf(k_s) - g_s - i_s)
\] \#\#Solución

A partir de la ecuación de euler se verifica para cada \(s \geq t\) y
cada \(j \geq s\) lo siguiente,

\[
\Delta_j c_j = \beta^{j-s}\Delta_sc_s
\] Por lo tanto, para cada \(s \geq t\) el consumo está dado por,

\[
c_s = (1-\beta)(1 + r_{s-1})b_s + \widehat{A_sf(k_s)} - \hat{g_s} - \hat{i_s}
\]

El lado derecho de la ecuación anterior es conocida pues a partir de la
condición de eficiencia del capital puede conocerce la senda de
equilibrio.

\subsection{Supuestos habituales}\label{supuestos-habituales-2}

Si \(r_{s-1} = r^*\) para cada \(s \geq t\) y \(\beta(1 + r^*) = 1\)
entonces, por la condición de euler, el consumo es el mismo en todos los
periodos y

\[
c = c_s = r^*(b_s - d_s) + \widehat{A_sf(k_s)} - \hat{g_s} - \hat{i_s}
\]

En este caso, la cuenta corriente está dada por,

\[
CA_s = (A_sf(k_s) - \widehat{A_sf(k_s)}) - (g_s - \hat{g_s})- (i_s - \hat{i_s})
\]

\subsection{Ejemplo}\label{ejemplo-3}

Supongamos que \(r_{s-1} = r^*\) para cada \(s \geq t\),
\(\beta(1 + r^*) = 1\), \(b_t = 0\), \(f(k_s) = k_s^{\alpha}\),
\(g_s = 0\), \(A_s = A\) para cada \(s \geq t\). Además se supone que el
capital inicial cumple con la condición de eficiencia del capital en
\(t\); es decir, \(\alpha Ak_t^{\alpha-1} = r^*\). Como \(A\) no cambia
en el tiempo, el capital es el mismo para todos los periodos pues la
condición de eficiencia señala que \(\alpha Ak_{s+1}^{\alpha -1} = r^*\)
para \(s \geq t+1\). Por lo tanto, \(k = k_{s}\) para cada \(s \geq t\).
Con lo anterior, \(i_s = 0\) para cada \(s \geq t\) y
\(A_s f(k_s) = Ak^{\alpha}\) de aquí se verifica que,

\[
c = c_t = A k_t ^{\alpha}
\]

\subsubsection{Choque en la
productividad}\label{choque-en-la-productividad}

Apartir de las condiciones anteriores, suponemos un choque positivo en
la productividad a partir del periodo \(t\) durante \(h\) periodos.
Formalmente, \(A_s = \lambda^{1-\alpha}A\) con \(\lambda > 1\) si
\(t \leq s \leq t+h-1\) y \(A_s = A\) si \(s \geq t+h\). Además
suponemos que el capital inicial cumple con la condición de eficiencia
del capital en asuencia de choque, es decir,
\(\alpha Ak_t^{\alpha-1} = r^*\). Definimos \(k_0 = k_t\). Luego,
\(k_s = k_0\) para \(s \geq t+h\) y \(k_s = k_1\) para
\(t \leq s \leq t+h-1\) donde
\(\alpha \lambda^{1-\alpha}Ak_1^{\alpha-1}\). Luego,

\[
\frac{\alpha Ak_0^{\alpha-1}}{\alpha \lambda^{1-\alpha}Ak_1^{\alpha-1}} = 1
\]

Así,a partir de lo anterior,

\[
(\lambda k_0)^{\alpha-1} = k_1^{\alpha-1}
\] Por lo que, \(k_1 = \lambda k_0\). A partir de esto tenemos tres
niveles de producción. Si \(s \geq t+h\) entonces,

\[
y_s = Ak_0^{\alpha} = y_0
\]

Por otro lado, si \(t+1 \leq s \leq t+h-1\) entonces

\[
y_s = \lambda^{1-\alpha}Ak_1^\alpha = \lambda Ak_0^{\alpha} = \lambda y_0 = y_1
\]

Por último, \(y_t = \lambda^{1-\alpha}y_0\). Con el valor de la
producción, tenemos que,

\[
\begin{align}
\hat{y_t} &= (1-\beta)\sum_{s\geq t}\beta^{s-t} y_s \\
&=(1-\beta)\left(\lambda^{1-\alpha}y_0 + \sum_{s = t+1}^{t+h-1}\beta^{s-t}\lambda y_0 + \sum_{s\geq t+h}\beta^{s-t}y_0 \right) \\
&= (1-\beta)\left(\lambda^{1-\alpha}y_0 + \frac{\beta(1-\beta^{h})\lambda y_0}{1-\beta} + \frac{\beta^h y_0}{1-\beta} \right) \\
&= (1-\beta)\lambda^{1-\alpha}y_0 + \beta(1-\beta^{h-1})\lambda y_0 + \beta^h y_0
\end{align}
\]

Asimismo, se verifica para la inversión que
\(i_t = k_1 - k_0 = (\lambda-1)k_0\), \(i_s = k_{s+1} - k_s = 0\) para
\(t + 1 \leq s \leq t+h-2\),
\(i_s = k_{s+1}-k_s = k_0 - k_1 = (1-\lambda)k_0\) si \(s = t+h-1\) y
por último, \(i_s = 0\) en otro caso. Así,

\[
\begin{align}
\hat{i_t} &= (1-\beta) \sum_{s\geq t} \beta^{s-t} i_s \\
&= (1-\beta)\left(i_t + \sum_{s = t+1}^{t+h-2}\beta^{s-t} i_s + \beta^{(t+h-1)-t}i_{t+h-1} + \sum_{s \geq t+h} \beta^{s-t} i_s \right) \\
&= (1-\beta)((\lambda-1)k_0 + \beta^{h-1}(1-\lambda)k_0) \\
&= (1-\beta)(1-\beta^{h-1})(\lambda - 1)k_0
\end{align}
\] Por lo tanto, el consumo cuando el choque permanece \(h\) periodos,
denotado por \(c_h\), está dado por,

\[
\begin{align}
c_h &= c_t \\
& = \hat{y_t} - \hat{i_t} \\
&= (1-\beta)\lambda^{1-\alpha}y_0 + \beta(1-\beta^{h-1})\lambda y_0 + \beta^h y_0 - (1-\beta)(1-\beta^{h-1})(\lambda - 1)k_0 
\end{align}
\] Definamos \(c_1\) como el consumo cuando el choque permanece un solo
periodo, es decir \(h = 1\). Luego,

\[
\begin{align}
c_1 &= (1-\beta)\lambda^{1-\alpha}y_0 + \beta(1-\beta^0)\lambda y_0 + \beta y_0 - (1-\beta)(1-\beta^{0})(\lambda - 1)k_0  \\
&= (1-\beta)\lambda^{1-\alpha} y_0 + \beta y_0
\end{align}
\]

Definamos, por otro lado, \(c_p\) como el consumo cuando el choque es
permanente, es decir \(h \rightarrow \infty\). Luego,

\[
\begin{align}
c_p &= \lim_{h \rightarrow \infty}  (1-\beta)\lambda^{1-\alpha}y_0 + \beta(1-\beta^{h-1})\lambda y_0 + \beta^h y_0 - (1-\beta)(1-\beta^{h-1})(\lambda - 1)k_0 \\
&= (1-\beta)\lambda^{1-\alpha}y_0 + \beta\lambda y_0 - (1-\beta)(\lambda - 1)k_0
\end{align}
\] Por lo tanto, se verifica que

\[0o, _.nh cx
c_h = (1-\beta^{h-1})c_p + \beta^{h-1}c_1
\]

\subsubsection{Términos de intercambio}\label{terminos-de-intercambio-1}

Ahora introducimoes los términos de intercambio
\(\{\Omega_s\}_{s\geq t}\). En este caso el producto en cada periodo
está determinado por \(\Omega_s A_s f(k_s)\), por lo que la solución es
similar a la anterior. Si suponemos que \(A_s = 1\) para cada
\(s \ geq t\) y que hay un choque positivo en los términos de
intercammbio, es decir, \(\Omega_s = \Omega\) si \(s \leq t+1\),
\(s \geq t+h\) y \(\Omega_s = \lambda^{1-\alpha}\Omega\) si
\(t \leq s \leq t+h-1\). En este caso, el choque es exactamente el mismo
que en el caso anterior.

\subsubsection{Senda de crecimiento
balanceado}\label{senda-de-crecimiento-balanceado}

Supongamos que la productividad crece a tasa constante, esto es

\[
\frac{A_{s+1}}{A_s} = 1 + \hat{A}
\]

Para cada \(s \geq t\). Además supongamos que la condición inicial del
capital satisface la condición de optimalidad del capital; es decir,
\(\alpha A_t k_t^{\alpha-1} = r^*\). Este supuesto una implicación
fundamental: el capital y la producción crecen a la misma tasa
constante. Para verlo, recordemos que para cada \(s \geq t+1\) se
verifica que en el óptimo, \(\alpha A_sk_s^{\alpha-1} = r^*\). A partir
del supuesto de la condición inicial del capital, se cumple que para
cada \(s \geq t\),

\[
\frac{\alpha A_s k_s^{\alpha-1}}{\alpha A_{s+1} k_{s+1}^{\alpha-1}} = 1
\]

Por lo que,

\[
\frac{k_{s+1}}{k_s} = (1 + \hat{A})^{\frac{1}{1-\alpha}} = 1 + \hat{y}
\]

La producción por otro lado,

\[
\frac{A_{s+1}k_{s+1}^{\alpha}}{A_s k_{s}^{\alpha}} = (1 + \hat{A})^{\frac{1}{1-\alpha}} = 1 + \hat{y}
\]

Lo anterior no es un supuesto del modelo sino un resultado de que la
productivdad crezca a una tasa constante. Por otro lado, supongamos que
\(1+ r_{s-1} = 1+r^* = \beta^{-1}(1+ \hat{y})\) para cada \(s \geq t\).
En este caso, a partir de la ecuación de euler se verifica que
\(c_{s+1} = (1+\hat{y})c_s\) para cada \(s \geq t\); es decir, el
consumo crece a la misma tasa constante que el producto y el capital.
Por tanto, para cualquier \(s \geq t\),

\[
c_s = (1 + \hat{y})^{s-t}c_t
\] Del mismo modo, para cada \(s \geq t\)

\[
y_s = (1 + \hat{y})^{s-t}y_t
\]

Y

\[
i_s = k_{s+1} - k_s = (1+\hat{y})k_s-k_s = (1+\hat{y})^{s-t}\hat{y}k_t = (1+\hat{y})^{s-t}i_t 
\]

Por lo tanto, para cada \(s \geq t\),

\[
y_s - i_s = (1 + \hat{y})^{s-t}(y_t-i_t)
\]

Utilizando la ecuación anterior y que
\(\frac{1+\hat{y}}{1+r^*} = \beta\) se tiene que,

\[
\begin{align}
\hat{y_t} -\hat{i_t} &= (1-\beta)\sum_{s \geq t} \left(\frac{1}{1+r^*}\right)^{s-t}(y_s - i_s) \\
&= (1-\beta)(y_t - i_t)\sum_{s\geq t} \left(\frac{1 + \hat{y}}{1+r^*}\right)^{s-t} \\
&= (1-\beta)(y_t - i_t)\sum_{s\geq t}\beta^{s-t} \\
&= y_t - i_t
\end{align}
\]

Por lo tanto,

\[
\begin{align}
c_t &= (1-\beta)(1+r^*)b_t + y_t - i_t\\
&=(r^*-\hat{y})b_t + y_t - i_t
\end{align}
\]

La balanza comercial está dada por,

\[
T_s = y_s - c_s - i_s = (1+\hat{y})^{s-t}(y_t-c_t-i_t) = (1+\hat{y})^{s-t}T_t
\]

Con

\[
T_t = y_t-c_t-i_t = y_t - i_t - (r^*-\hat{y})b_t - (y_t - i_t) = -(r^*-\hat{y})b_t
\]

La cuenta corriente está dada por,

\[
CA_t = T_t + r^*b_t =  -(r^*-\hat{y})b_t + r^* b_t = \hat{y}b_t
\]

Como además \(CA_t = b_{t+1} - b_t = \hat{y} b_t\) entonces
\(b_{s+1} = (1+\hat{y})b_s\) para cada \(s \geq t\). De donde se obtiene
que

\[
CA_s = b_{s+1} - b_s = \hat{y}b_s = (1+\hat{y})^{s-t}\hat{y}b_t = (1+\hat{y})^{s-t}CA_t
\]

En particular,

\[
\frac{CA_s}{y_s} = \hat{y} \frac{b_t}{y_t}
\]

\paragraph{Depreciación del capital}\label{depreciacion-del-capital}

A partir de las condciones anteriores, supongamos ahora que el capital
sufre un choque negativo e inesperado en \(t\). Formalmente,
\(i_s =(\theta + \hat{y})k_s\) si \(s = t\) con \(\theta \in (0,1)\) y
\(i_s = (1 + \hat{y})k_s\) para \(s \geq t+1\).

\section{Modelo con dos bienes,
dotación}\label{modelo-con-dos-bienes-dotacion}

En este caso, la economía posee dos bienes: los comerciales, que
denotamos por \(T\) y los no comerciables que denotamos por \(n\). De
esta forma, para cada periodo \(s \geq t\) el consumo de bienes
comerciales en el periodo \(s\) se denota con \(c_{Ts}\) y el consumo de
bienes no comerciables en el mismo periodo se denota por \(c_{ns}\).
Además, se define un índice de consumo \(c_s\) para cada periodo
\(s \geq t\) dado por \(c_s = c_{Ts}^\gamma c_{ns}^{1-\gamma}\) y
\(p_s\) como el precio relativo del bien comerciable en términos del no
comerciable.

\subsection{Problema de optimización}\label{problema-de-optimizacion-4}

El problema de oprimización del agente representativo es

\[
\begin{align}
\max_{c_{Ts}, c_{ns}, b_{s+1}, s\geq t}  \sum_{s\geq t} \ln(c_s) \\
 c_s &= c_{Ts}^\gamma c_{ns}^{1-\gamma} \\
c_{Ts} + p_sc_{ns} + b_{t+1} &= y_{Ts} + p_sy_{ns} + (1 + r_{s-1})b_s
\end{align}
\]

En este problema, están dados \(\{y_{Ts}, y_{ns}, r_{s-1}\}_{s\geq t}\)
y \(b_t\). En este caso, \(c_{ns} = y_{ns}\) para cada \(s \geq t\) ya
que estos bienes no se comercian. Por lo tanto, la balanza comercial
está dada por,

\[
T_s = y_{T_s} - c_{Ts}
\] Para cada \(s\geq t\). Y la cuenta corriente está dada por,

\[
CA_s = b_{s+1} - b_s = y_{T_s} - c_{Ts} + r_{s-1}b_s = T_s + r_{s-1}b_s
\] Para cada \(s \geq t\). Además, este problema puede dividirse en dos,
como lo veremos a continuación.

\subsubsection{Problema de optimización
intratemporal}\label{problema-de-optimizacion-intratemporal}

Supongamos que \(c_s\) y \(p_s\) están fijos para alguna \(s \geq t\)
arbitraria y fija. Luego, el agente representativo busca encontrar la
combinación de bienes que minimice su gasto \(c_{Ts} + p_s c_{ns}\). Es
decir, el agente busca,

\[
\min_{c_{Ts}, c_{ns}}\{c_{Ts} + p_sc_{ns}: c_s = c_{Ts}^\gamma c_{ns}^{1-\gamma}\}
\]

A partir de la solución, se obtiene la función de gasto óptimo que toma
como argumentos \(p_s\) y \(c_s\), esto es

\[
e_s = e(p_s,c_s) = \min_{c_{Ts}, c_{ns}}\{c_{Ts} + p_sc_{ns}: c_s = c_{Ts}^\gamma c_{ns}^{1-\gamma}\}
\]

En particular, definimos \(P_s = e(p_s, 1)\) para cada \(s \geq t\), el
gasto óptimo cuando el índice de consumo es unitario. Dado que el índice
es CD, se cumple que \(P_s c_s = e(p_s, c_s)\).

\subsubsection{Problema de optimización
intertemporal}\label{problema-de-optimizacion-intertemporal}

Con el problema anterior, ahora el objetivo intertemporal del agente
representativo es

\[
\begin{align}
\max_{c_s, s \geq t} \sum_{s \geq t} \ln(c_s) \\
& P_s c_s + b_{t+1} = y_{Ts} + p_s y_{ns} + (1 + r_{s-1})b_s
\end{align}
\]

\subsection{Condiciones de primer
orden}\label{condiciones-de-primer-orden-4}

\subsubsection{Intratemporal}\label{intratemporal}

A partir del problema intertemporal se obtiene que, en equilibrio, para
cada \(s \geq t\),

\[
p_s = \frac{\gamma c_{Ts}}{(1-\gamma)c_{ns}}
\]

Además, con esto se verifica que

\[
P_s = \frac{1}{\gamma}\left(\frac{\gamma}{1-\gamma}p_s\right)^{1-\gamma }
\]

Con lo anterior, se cumple que \(c_{Ts} = \gamma P_s c_s\) y
\(c_{ns} = (1-\gamma)P_sc_s\).

\subsubsection{Intertemporal}\label{intertemporal}

La condición de optimalidad intertemporal obtenida del problema
respectivo es, para cada \(s \geq t\),

\[
\begin{equation}
P_{s+1}c_{s+1} = \beta(1 + r_s)P_sc_s
\label{eq:optdosb}
\end{equation}
\]

Utilizando el valor de \(P_s\) de la sección anterior, obtenemos que,

\[
c_{s+1} = \beta(1+r_s)\left(\frac{p_s}{p_{s+1}}\right)^{1 - \gamma}c_s
\]

\subsubsection{Relación entre las condiciones de
optimalidad}\label{relacion-entre-las-condiciones-de-optimalidad}

ES conveniente identificar la relación exitente entre las condiciones de
optimalidad previas. Recordemos que de la condición de eficiencia
intratemporal \(c_{Ts} = \gamma P_sc_s\). Por lo tanto, sustituyendo en
\eqref{eq:optdosb} se tiene que para cada \(s \geq t\),

\[
c_{T(s+1)} = \beta(1 +r_{s})c_{Ts}
\]

Recuperamos la ecuación de euler para los bienes comerciables.

\subsection{Restricción presupuestal}\label{restriccion-presupuestal-4}

Dado que \(n\) no se comercian, se debe cumplir la siguiente condición
de vaciado para cada \(s \geq t\),

\[
c_{ns} = y_{ns}
\]

Por lo tanto, en equilibrio se obtiene que, para cada \(s \geq t\),

\[
\sum_{j \geq s} \Delta_j T_j = -(1+r_{s-1})b_s
\] O bien,

\[
\sum_{j \geq s}\Delta_j c_{Tj}= (1 + r_{s-1})b_s + \sum_{j\geq s}\Delta_j y_{Tj}
\]

\subsection{Solución}\label{solucion-3}

Dado que la ecuación de euler se sigue cumpliendo para los bienes
comerciables, entonces para cada \(s \geq t\), en equilibrio,

\[
c_{Ts} = (1-\beta)(1+r_{s-1})b_s + \widehat{y_{Ts}}
\] Donde el lado derecho son parámetros conocidos. Además, por la
condición de vaciado, para cada \(s \geq t\),

\[
c_{ns} = y_{ns}
\]

Donde el lado derecho son parámetros conocidos. Por último, el precio en
cada periodo \(s \geq t\) queda determinado por,

\[
p_s  = \frac{\gamma c_{Ts}}{(1 - \gamma)c_{ns}}
\]

\subsection{Supuestos habituales}\label{supuestos-habituales-3}

Si \(r_{s-1} = r^*\) para cada \(s \geq t\) y \(\beta(1 + r^*) = 1\)
entonces, por la condición de euler, el consumo de bienes no
comerciables es el mismo en todos los periodos y

\[
c = c_s = r^*b_s + \widehat{y_{Ts}} 
\]

En este caso, la cuenta corriente está dada por,

\[
CA_s = y_{Ts} - \widehat{y_{Ts}}
\]

\subsection{Ejemplo}\label{ejemplo-4}

Supongamos que \(r_{s-1} = r^*\) para cada \(s \geq t\) y
\(\beta(1 + r^*) = 1\), \(y_{Ts} = y_T\), \(y_{ns} = y_n\) para cada
\(s \geq t\) y además que \(b_t = 0\). En este caso, el consumo es el
mismo para ambos bienes en todos los periodos y dado que la dotación es
constante se verfica que \(c_{Ts} = c_T = y_T\), \(c_{ns} = c_n = y_n\).
Con esto,

\[
p_s = p = \frac{\gamma c_T}{(1- \gamma)c_n} = \frac{\gamma y_T}{(1- \gamma)y_n}
\] Finalmente, \(T_s = CA_s = 0\), para cada \(s \geq t\).

\subsubsection{Choque permanente en bienes
comerciables}\label{choque-permanente-en-bienes-comerciables}

Supongamos ahora que \(y_{Ts} = \lambda y_T\) para cada \(s \geq t\).
Denotamos como \(c_T^p\) el consumo en este choque. Claramente
\(c_T^p = \lambda y_T\).

\subsubsection{Choque de un periodo en bienes
comerciables}\label{choque-de-un-periodo-en-bienes-comerciables}

A partir de la situación anterior supongamos que hay un choque negativo
en la dotación de bienes comerciables. Formalmente,
\(y_{Ts} = \lambda y_{T}\) si \(s = t\) y \(y_{Ts} = y_T\) si
\(s \geq t\). El consumo de bienes no comerciables sigue siendo el
mismo. Además se sigue sosteniendo que el consumo de bienes comerciables
es el mismo en cada periodo por lo que. Con esta idea en mente, notemos
que,

\[
\begin{align}
\widehat{y_{Tt}} &= (1-\beta)\sum_{s\geq t}\beta^{s-t}y_{Ts} \\
&= (1-\beta)\left(\lambda y_T + \sum_{s\geq t+1}\beta^{s-t}y_{T}\right) \\
&= (1-\beta)\left(\lambda y_T + \frac{\beta y_{T}}{1-\beta}\right) \\
&= (1-\beta)\lambda y_T + \beta y_T
\end{align}
\]

Por lo tanto, si denotamos el consumo en este choque como \(c_T^1\), se
verfica que,

\[
c_{Ts}^1 = c_T^1 = \widehat{y_{Tt}} = (1-\beta)\lambda y_T + \beta y_T < y_T
\]

\section{Modelo con dos bienes, dotación y
gobierno}\label{modelo-con-dos-bienes-dotacion-y-gobierno}

\subsection{Gobierno}\label{gobierno-1}

En este caso introducimos un gobierno que consume ambos bienes.
Denotamos con \(g_{Ts}\) la cantidad de bienes comerciables que consume
el gobierno en el periodo \(s\) y equivalentemente \(g_{ns}\) la
cantidad de bienes no comerciables que consume el gobierno en el periodo
\(s\). Luego, el gobierno debe satisfacer la siguiente restricción
presupuestal,

\[
g_{Ts} + p_s g_{ns} = p_s \tau_s + b_{s+1} - (1 + r_{s-1})b_s
\] Para cada \(s \geq t\). Iterando hacia adelante a partir del periodo
\(s\) se obtiene que

\[
\sum_{j\geq s} \Delta_j (g_{Tj}+p_jg_{nj}) = -(1+r_{s-1})b_s + \sum_{j \geq s} \Delta_j p_j \tau_j
\]

\subsection{Problema de optimización}\label{problema-de-optimizacion-5}

En este caso, el agente representativo busca,

\[
\begin{align}
\max_{c_{Ts}, c_{ns}, b_{s+1}, s\geq t}  \sum_{s\geq t}(\ln(c_s) + \ln(g_s)) \\
 c_s &= c_{Ts}^\gamma c_{ns}^{1-\gamma} \\
c_{Ts} + p_sc_{ns} + p_s \tau_s + b_{t+1} &= y_{Ts} + p_sy_{ns} + (1 + r_{s-1})b_s
\end{align}
\]

En este caso están dados
\(\{y_{Ts}, y_{ns}, g_{Ts}, g_{ns}, r_{s-1}\}_{s\geq t}\), \(d_t\) y
\(b_T\). Sustituyendo la restricción presupuestal del gobierno en la
restricción del agente representativo obtenemos que,

\[
c_{Ts} + p_s c_{ns} + g_{Ts} + p_s g_{ns}+b_{t+1}-d_{t+1} = y_{Ts} + p_sy_{ns}+(1+r_{s-1})(b_s-d_s)
\] En este caso la condición de vaciado del mercado de bienes no
comerciables es

\[
c_{ns} = y_{ns} - g_{ns}
\] Por lo que la balanza comercial está dada por,

\[
T_s = y_{T_s} - c_{Ts}-g_{Ts}
\] Y la cuenta corriente está dada por,

\[
CA_s = (b_{s+1}-d_{s+1})-(b_s-d_s) = T_s + r_{s-1}(b_s-d_s)  = y_{T_s} - c_{Ts}-g_{Ts} + r_{s-1}(b_s-d_s)
\]

\subsection{Condiciones de primer
orden}\label{condiciones-de-primer-orden-5}

Las condiciones de optimalidad no sufren ninguna modificación dado que
se trata de un impuesto de suma fija. De este modo, se tiene que para
cualquier \(s \geq t\),

\[
c_{T(s+1)} = \beta(1+r_{s-1})c_{Ts}
\] Además,

\[
p_s = \frac{\gamma c_{Ts}}{(1-\gamma)c_{ns}}
\] Y la condición de vaciado,

\[
c_{ns} = y_{ns} - g_{ns}
\]

Para cada \(s \geq t\).

\subsection{Restricción presupuestal}\label{restriccion-presupuestal-5}

Considerando la condición de vaciado de mercado e iterando la
restricción presupuestal hacia adeltante, resulta que para cada
\(s\geq t\),

\[
\sum_{j\geq s}\Delta_jT_j = -(1+r_{s-1})(b_s-d_s)
\]

O bien,

\[
\sum_{j\geq s}\Delta_j c_{Tj} = (1+r_{s-1})(b_s-d_s) + \sum_{j \geq s}\Delta_j (y_{Tj} - g_{Tj})
\]

\subsection{Solución}\label{solucion-4}

A partir de la ecuación de euler para los bienes comerciables se obtiene
que para cada \(s \geq t\) y \(j \geq s\),

\[
\Delta_j c_{Tj} = \beta^{s-j}c_{Ts}
\]

Por lo tanto, para cada \(s \geq t\),

\[
c_{Ts} = (1-\beta)(1+r_{s-1})(b_s - d_s) + \widehat{y_{Ts}} - \widehat{g_{Tj}}
\] \#\# Ejemplo

\subsubsection{Dotación del gobierno de bienes
comerciables}\label{dotacion-del-gobierno-de-bienes-comerciables}

Hay un caso particular en el que el gobierno posee una dotación de
bienes comerciables en todos los periodos, \(\{y_s^g\}_{s\geq t}\). Esto
se agrega al modelo con \(y_s^g = -g_{Ts}\) para cada \(s \geq t\) y
considerando la senda de impuestos \(\{\tau_s\}_{s\geq t}\) como
exógena.

\section{Exámenes de parciales}\label{examenes-de-parciales}

En esta sección se elaboran distintos exámenes parciales de la clase.

\subsection{Examen Octubre 2015}\label{examen-octubre-2015}

\subsection{Pregunta 1}\label{pregunta-1}

Consideremos una economía pequeña y abierta con capital en la que
\(\beta(1+r_{s-1}) = \beta(1+r^*)\) para cada \(s \geq t\). Supongamos
que el capital en el periodo \(t\) satisface que \(Af'(k_t) = r^*\) y
además los agentes esperan que la productividad aumenté a \(\lambda A\),
con \(\lambda > 1\) para \(t+1 \leq s \leq t+h\), regresando a \(A\) a
partir \(t + h -1\), ¿el valor de \(b_{t+h+1}\) es mayor entre ,ás
grande sea h?

\subsubsection{Solución}\label{solucion-5}

Como hay dos niveles de productividad, conforme a la condición de
optimalidad del capital, habrá dos niveles de capital óptimo. Definimos
\(k_0\) tal que \(Af'(k_0) = r^*\) y \(k_1\) tal que
\(\lambda Af'(k_1) = r^*\). Luego, definiendo \(f_0 = f(k_0)\) y
\(f_1 = f(k_1)\), tenemos que

\begin{longtable}[]{@{}lllll@{}}
\toprule
\(s\) & \(A_s\) & \(k_s\) & \(y_s\) & \(i_s\)\tabularnewline
\midrule
\endhead
\(t\) & \(A\) & \(k_0\) & \(Af_0\) & \(k_{t+1}-k_t = 0\)\tabularnewline
\(t+1\) & \(\lambda A\) & \(k_0\) & \(\lambda Af_0\) &
\(k_{t+2}-k_{t+1} = k_1 - k_0\)\tabularnewline
\(t+2\) & \(\lambda A\) & \(k_1\) & \(\lambda Af_1\) &
\(k_{t+3}-k_{t+2} = 0\)\tabularnewline
\(t+h\) & \(\lambda A\) & \(k_1\) & \(\lambda Af_1\) &
\(k_{t+h+1} - k_{t+h} = k_0 - k_1\)\tabularnewline
\(t + h + 1\) & \(A\) & \(k_0\) & \(Af_0\) &
\(k_{t+h+2}-k_{t+h+1} = 0\)\tabularnewline
\bottomrule
\end{longtable}

Además dado que se cumple la ecuación de euler, el consumo \(c\) es el
mismo para todos los periodos. Con esto, por un lado,

\[
c = r^*b_t + \hat{y_t} - \hat{i_t}
\]

Y por otro lado,

\[
c = r^* b_{t+h+1} + \widehat{y_{t+h+1}} - \widehat{i_{t+h+1}}
\] Por lo tanto,

\[
b_{t+h+1} = b_t + \frac{1}{r^*}(\hat{y_t} - \widehat{y_{t+h+1}} - (\hat{i_t} - \widehat{i_{t+h+1}}))
\] Claramente \(\widehat{y_{t+h+1}} = Af_0\) pues a partir de \(t+h+1\)
la situación vuelve a la normalidad. Lo mismo sucede con
\(\widehat{i_{t+h+1}} = 0\). Por otro lado,

\[
\begin{align}
\hat{y_t} &= (1-\beta)\sum_{s\geq t}\beta^{s-t}y_s \\
& = (1-\beta)\left(Af_0 + \beta \lambda A f_0 + \sum_{s = t+2}^{t+h}\beta^{s-t}\lambda Af_1 + \sum_{s\geq t+h+1}\beta^{s-t} Af_0\right) \\
&= (1-\beta)\left((1+\beta \lambda)Af_0 + \frac{\beta^2(1-\beta^{h-1})\lambda Af_1}{1-\beta} + \frac{\beta^{h+1}Af_0}{1-\beta}\right) \\
&= ((1-\beta)(1+\beta \lambda) + \beta^{h+1})Af_0 + \beta^2(1-\beta^{h-1})\lambda Af_1
\end{align}
\]

Para la inversión,

\[
\begin{align}
\hat{i_t} &= (1-\beta)\sum_{s\geq t} \beta^{s-t} i_s \\
&=(1-\beta)(\beta(k_1-k_0) + \beta^h (k_0-k_1)) = (1-\beta)\beta(1-\beta^h)(k_1 - k_0)
\end{align}
\] Obsérvese que dados los rendimientos decrecientes a escala de \(f\)
se verfifica que \(k_1 - k_0 > 0\) por lo que \(\hat{i_t}<0\). Así,

\[
b_{t+h+1} = b_t + \frac{1}{r^*}(((1-\beta)(1+\beta \lambda) + \beta^{h+1})Af_0 + \beta^2(1-\beta^{h-1})\lambda Af_1 - Af_0- (1-\beta)\beta(1-\beta^h)(k_1 - k_0) )
\]

\subsection{Problema 2}\label{problema-2}

Se tiene una economía en la trayectoria de crecimiento balanceado
\(\frac{A_{S+1}}{A_s} = 1 + \hat{A}\) para cada \(s \geq t\), con
\(b_t = 0\) y se da una destrucción parcial e inesperada de su acervo de
capital en el periodo \(t\). ¿Eveltualmente la cuenta corriente como la
balanza comercial serán negativas?


\end{document}
